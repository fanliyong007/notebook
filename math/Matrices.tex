\chapter{矩阵}
\section{矩阵}
\subsection{矩阵的概念}
{\color{red}定义:}有$m\times n$个数$a_{ij}(i=1,2,\cdots,m;j=1,2,\cdots,n)$排成的一个$m$行$n$列矩形数表,称为
一个${\color{blue}m\times n}$矩阵,记作
$$
A_{m\times n}=A=
\left[ \begin{matrix}
    {{a}_{11}} & {{a}_{12}} & \cdots  & {{a}_{1n}}  \\
    {{a}_{21}} & {{a}_{22}} & \cdots  & {{a}_{2n}}  \\
    \vdots  & \vdots  & {\ddots} & \vdots   \\
    {{a}_{m1}} & {{a}_{m2}} & \cdots  & {{a}_{mn}}  \\
 \end{matrix} \right] 
=(a_{ij})_{m\times n}
$$
称$a_{ij}$为这个矩阵的第$i$行第$j$列的元素,称$m{\color{red}\times} n$(即:行数$\times$列数)为该{\color{blue}矩阵的型}\\
\subsection{矩阵的相等}
\begin{enumerate}
    \item{如果两个矩阵的型相同, 称为同型矩阵。}
    \item{
        两个矩阵$A=(a_{ij})$与$B=(b_{ij})$为同型矩阵, 且对应元素均相等,即
        $$
        a_{ij}=b_{ij}(i=1,2,\cdots,m;j=1,2,\cdots,n)
        $$
        则称矩阵A和B相等,记作A=B
        }
\end{enumerate}
\subsection{特殊矩阵}
\begin{enumerate}
    \item{行矩阵(或行向量):只有一行的矩阵称为行矩阵(或行向量)。\\
    $$A=\left[\begin{matrix}
        a_1\quad a_2\quad\cdots\quad a_n
    \end{matrix}
    \right]
    =(a_1,a_2,\cdots,a_n)$$
    }
    \item{列矩阵(或列向量):只有一列的矩阵称为列矩阵(或列向量)
        $$B=\left[\begin{matrix}
            b_1 \\ b_2 \\ \vdots \\ b_n
        \end{matrix}
            \right]
        $$
    }
    \item{
        方阵:如果矩阵$A=(a_{ij})$的行数与列数都等于$n$, 则称$A$为$n$阶矩阵(或称$n$阶方阵)。
        $$
        A=\left[\begin{matrix}
            {{a}_{11}} & {{a}_{12}} & \cdots  & {{a}_{1n}}  \\
            {{a}_{21}} & {{a}_{22}} & \cdots  & {{a}_{2n}}  \\
            \vdots  & \vdots  & {\ddots} & \vdots   \\
            {{a}_{n1}} & {{a}_{n2}} & \cdots  & {{a}_{nn}}  \\
         \end{matrix} \right]         
        $$
        {\color{red}注\quad}矩阵与行列式有本质区别:\\
        行列式是表示一个数值, 而矩阵仅表示数表本身。矩阵的行数和列数可以不同,符号不同。
    }
    \item{
        零矩阵:元素全为零的矩阵称为零矩阵。$m\times n$零矩阵记作$0_{m\times n}$或$0$。\\
        {\color{red}注\quad}不同型的零矩阵是不相等的
        $$
        \left[\begin{matrix}
            0 & 0 & 0 & 0  \\
            0 & 0 & 0 & 0  \\
            0 & 0 & 0 & 0  \\
            0 & 0 & 0 & 0  \\
         \end{matrix} \right]\ne 
         \left[ \begin{matrix}
            0 & 0 & 0 & 0  \\
         \end{matrix} \right]         
        $$
    }
    \item{
        上三角形矩阵和下三角形矩阵:称形如
        $$
        \left[ \begin{matrix}
            {{a}_{11}} & {{a}_{12}} & \cdots  & {{a}_{1n}}  \\
            0 & {{a}_{22}} & \cdots  & {{a}_{2n}}  \\
            \vdots  & \vdots  & {\ddots} & \vdots   \\
            0 & 0 & \cdots  & {{a}_{nn}}  \\
         \end{matrix} \right]         
        $$
        的矩阵为上三角矩阵。\\
        {\color{blue}性质\quad}如果矩阵$A={a_{ij}}$是上三角矩阵,则当$i>j$时,必有$a_{ij}=0$。
        类似的,称形如
         $$
         \left[ \begin{matrix}
            {{a}_{11}} & 0 & 0 & 0  \\
            {{a}_{21}} & {{a}_{22}} & \cdots  & 0  \\
            \vdots  & \vdots  & {\ddots} & \vdots   \\
            {{a}_{n1}} & {{a}_{n2}} & \cdots  & {{a}_{nn}}  \\
         \end{matrix} \right]         
         $$
        的矩阵为下三角矩阵。
    }
    \item{
        对角矩阵,称形如
        $$
        \left[ \begin{matrix}
            \lambda_{1} & 0 & \cdots  & 0  \\
            0 & \lambda_{2} & \cdots  & 0  \\
            \vdots  & \vdots  & {} & \vdots   \\
            0 & 0 & \cdots  & \lambda_{n}  \\
        \end{matrix} \right]
        $$
        的矩阵为对角矩阵,可记作
         $A=diag(\lambda_{1},\lambda_{2},\cdots,\lambda_{n})$ \\
        {\color{blue}性质\quad}如果矩阵$A=({a}_{ij})$是对角矩阵,则当$i\neq j$时,必有$a_{ij}=0$
        {\color{red}注\quad}对角矩阵既是上三角矩阵, 也是下三角矩阵
    }
    \item{
        数量矩阵,单位矩阵 \\
        称形如
        $$
        \left[ \begin{matrix}
            \lambda  & 0 & \cdots  & 0  \\
            0 & \lambda  & \cdots  & 0  \\
            \vdots  & \vdots  & {} & \vdots   \\
            0 & 0 & \cdots  & \lambda   \\         
        \end{matrix} \right]
        $$
        的矩阵为{\color{blue}数量矩阵\quad} \\
        称
        $$
        \left[ \begin{matrix}
            1 & 0 & \cdots  & 0  \\
            0 & 1 & \cdots  & 0  \\
            \vdots  & \vdots  & {} & \vdots   \\
            0 & 0 & \cdots  & 1  \\
        \end{matrix} \right]
        $$
        为$n$阶{\color{blue}单位矩阵\quad}记为$E_n$或$E$
    }
\end{enumerate}
\subsection{矩阵的基本运算}
{\color{red}定义:}设$A=(a_{ij},B=(b_{ij}))$是两个$m\times n$矩阵(即同型),则记举证A与B的和为$A+B$规定
$$
A+B=\left[ \begin{matrix}
    {{a}_{11}}+{{b}_{11}} & {{a}_{12}}+{{b}_{12}} & \cdots  & {{a}_{1n}}+{{b}_{1n}}  \\
   {{a}_{21}}+{{b}_{21}} & {{a}_{22}}+{{b}_{22}} & \cdots  & {{a}_{2n}}+{{b}_{2n}}  \\
   \vdots  & \vdots  & {\ddots} & \vdots   \\
   {{a}_{m1}}+{{b}_{m1}} & {{a}_{m2}}+{{b}_{m2}} & \cdots  & {{a}_{mn}}+{{b}_{mn}}  \\      
\end{matrix} \right]
$$
即$A+B$的每个元素就是A与B的对应元素相加\\
{\color{red}注:\quad}仅当A与B为同型矩阵时,$A+B$才有定义\\
矩阵加法的运算规律
\begin{enumerate}
    \item{
        $A+B=B+A$
    }
    \item{
        $(A+B)+C=A+(B+C)$
    }
    \item {
        $A+0=0+A=A$
    }
\end{enumerate}
设矩阵$A=(a_{ij})_{m\times n}$\\
称
$$
\left[ \begin{matrix}
    -{{a}_{11}} & -{{a}_{12}} & \cdots  &-{{a}_{1n}}  \\
   -{{a}_{21}} & -{{a}_{22}} & \cdots  &-{{a}_{2n}}  \\
   \vdots  & \vdots  & \ddots & \vdots   \\
   -{{a}_{m1}} & -{{a}_{m1}} & \cdots  &-{{a}_{mn}}  \\    
\end{matrix} \right]
$$
为矩阵A的负矩阵,记为$-A$ \\
定义矩阵的减法:$A-B=A+(-B)$\\
{\color{red}注:\quad}仅当A与B为同型矩阵时,$A-B$才有定义\\
\subsubsection{矩阵的数量乘法}
设A时一个矩阵,k时一个数,则数量乘法$kA$也是一个矩阵,定义为
$$kA=
\left[ \begin{matrix}
    \text{{\color{red}k}}{{a}_{11}} & \text{{\color{red}k}}{{a}_{12}} & \cdots  & \text{{\color{red}k}}{{a}_{1n}}  \\
    \text{{\color{red}k}}{{a}_{21}} & \text{{\color{red}k}}{{a}_{22}} & \cdots  & \text{{\color{red}k}}{{a}_{2n}}  \\
    \vdots  & \vdots  & {\ddots} & \vdots   \\
    \text{{\color{red}k}}{{a}_{m1}} & \text{{\color{red}k}}{{a}_{m2}} & \cdots  & \text{{\color{red}k}}{{a}_{mn}}  \\   
\end{matrix} \right]
$$
数乘矩阵的运算规律:\\
\begin{enumerate}
    \item{
        $k(A+B)=kA+kB$
    }
    \item{
        $(k+l)A=kA+kB$
    }
    \item{
        $k(lA)=(kl)A$
    }
\end{enumerate}
加法和数乘合称为矩阵的{\color{blue}线性运算}\\
{\color{blue}性质}
\begin{enumerate}
    \item{
        $1A=A,{\color{red}0}A=0,(-1)A=-A$
    }
    \item{
        数量矩阵
        $$
        \left[ \begin{matrix}
            \lambda  & 0 & \cdots  & 0  \\
            0 & \lambda  & \cdots  & 0  \\
            \vdots  & \vdots  & {} & \vdots   \\
            0 & 0 & \cdots  & \lambda   \\         
        \end{matrix} \right]=\lambda E
        $$
    }
\end{enumerate}
\subsubsection{矩阵的乘法}
\subsubsection{矩阵的转置}
\subsubsection{方阵的行列式}
