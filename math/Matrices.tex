\chapter{矩阵}
\section{矩阵}
\subsection{矩阵的概念}
{\color{red}定义:}有$m\times n$个数$a_{ij}(i=1,2,\cdots,m;j=1,2,\cdots,n)$排成的一个$m$行$n$列矩形数表,称为
一个${\color{blue}m\times n}$矩阵,记作
$$
A_{m\times n}=A=
\left[ \begin{matrix}
    {{a}_{11}} & {{a}_{12}} & \cdots  & {{a}_{1n}}  \\
    {{a}_{21}} & {{a}_{22}} & \cdots  & {{a}_{2n}}  \\
    \vdots  & \vdots  & {\ddots} & \vdots   \\
    {{a}_{m1}} & {{a}_{m2}} & \cdots  & {{a}_{mn}}  \\
 \end{matrix} \right] 
=(a_{ij})_{m\times n}
$$
称$a_{ij}$为这个矩阵的第$i$行第$j$列的元素,称$m{\color{red}\times} n$(即:行数$\times$列数)为该{\color{blue}矩阵的型}\\
\subsection{矩阵的相等}
\begin{enumerate}
    \item{如果两个矩阵的型相同, 称为同型矩阵。}
    \item{
        两个矩阵$A=(a_{ij})$与$B=(b_{ij})$为同型矩阵, 且对应元素均相等,即
        $$
        a_{ij}=b_{ij}(i=1,2,\cdots,m;j=1,2,\cdots,n)
        $$
        则称矩阵A和B相等,记作A=B
        }
\end{enumerate}
\subsection{特殊矩阵}
\begin{enumerate}
    \item{行矩阵(或行向量):只有一行的矩阵称为行矩阵(或行向量)。\\
    $$A=\left[\begin{matrix}
        a_1\quad a_2\quad\cdots\quad a_n
    \end{matrix}
    \right]
    =(a_1,a_2,\cdots,a_n)$$
    }
    \item{列矩阵(或列向量):只有一列的矩阵称为列矩阵(或列向量)
        $$B=\left[\begin{matrix}
            b_1 \\ b_2 \\ \vdots \\ b_n
        \end{matrix}
            \right]
        $$
    }
    \item{
        方阵:如果矩阵$A=(a_{ij})$的行数与列数都等于$n$, 则称$A$为$n$阶矩阵(或称$n$阶方阵)。
        $$
        A=\left[\begin{matrix}
            {{a}_{11}} & {{a}_{12}} & \cdots  & {{a}_{1n}}  \\
            {{a}_{21}} & {{a}_{22}} & \cdots  & {{a}_{2n}}  \\
            \vdots  & \vdots  & {\ddots} & \vdots   \\
            {{a}_{n1}} & {{a}_{n2}} & \cdots  & {{a}_{nn}}  \\
         \end{matrix} \right]         
        $$
        {\color{red}注\quad}矩阵与行列式有本质区别:\\
        行列式是表示一个数值, 而矩阵仅表示数表本身。矩阵的行数和列数可以不同,符号不同。
    }
    \item{
        零矩阵:元素全为零的矩阵称为零矩阵。$m\times n$零矩阵记作$0_{m\times n}$或$0$。\\
        {\color{red}注\quad}不同型的零矩阵是不相等的
        $$
        \left[\begin{matrix}
            0 & 0 & 0 & 0  \\
            0 & 0 & 0 & 0  \\
            0 & 0 & 0 & 0  \\
            0 & 0 & 0 & 0  \\
         \end{matrix} \right]\ne 
         \left[ \begin{matrix}
            0 & 0 & 0 & 0  \\
         \end{matrix} \right]         
        $$
    }
    \item{
        上三角形矩阵和下三角形矩阵:称形如
        $$
        \left[ \begin{matrix}
            {{a}_{11}} & {{a}_{12}} & \cdots  & {{a}_{1n}}  \\
            0 & {{a}_{22}} & \cdots  & {{a}_{2n}}  \\
            \vdots  & \vdots  & {\ddots} & \vdots   \\
            0 & 0 & \cdots  & {{a}_{nn}}  \\
         \end{matrix} \right]         
        $$
        的矩阵为上三角矩阵。\\
        {\color{blue}性质\quad}如果矩阵$A={a_{ij}}$是上三角矩阵,则当$i>j$时,必有$a_{ij}=0$。
        类似的,称形如
         $$
         \left[ \begin{matrix}
            {{a}_{11}} & 0 & 0 & 0  \\
            {{a}_{21}} & {{a}_{22}} & \cdots  & 0  \\
            \vdots  & \vdots  & {\ddots} & \vdots   \\
            {{a}_{n1}} & {{a}_{n2}} & \cdots  & {{a}_{nn}}  \\
         \end{matrix} \right]         
         $$
        的矩阵为下三角矩阵。
    }
\end{enumerate}