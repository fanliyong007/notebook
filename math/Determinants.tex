\chapter{行列式}
\section{行列式的定义}
\subsection{二阶行列式}
定义一个
$$
\left|\begin{array}{cc} a_{11} & a_{12} \\
a_{21} & a_{22} \end{array}\right|
$$
为二阶行列式,主对角线为$a_{11}$和$a_{22}$,副对角线为$a_{12}$和$a_{21}$。并规定
$$
\left|\begin{array}{cc} a_{11} & a_{12} \\
a_{21} & a_{22} \end{array}\right|=a_{11}\times a_{22}-a_{12}\times a_{21}
$$
值相等的两个行列式称为这两个行列式相等。\\
\subsection{n 阶行列式}
称用$n^2$个元素$a_{ij}  (i,j =1,2,…,n)$ 组成的如下对象
$$
D=\left|
\begin{array}{cccc} 
    a_{11} & a_{12} & \cdots & a_{1n} \\ 
    a_{21} & a_{22} & \cdots & a_{2n} \\ 
    \vdots & \vdots & \ddots & \vdots \\
    a_{n1} & a_{n2} & \cdots & a_{nn} 
\end{array}
\right|=\det(a_{ij})=|a_{ij}|_{n}
$$
为一个n 阶行列式(determinant)。
$$
D=\left|
\begin{array}{cccc} 
    a_{11} & a_{12} & a_{13} & a_{14} \\
    a_{21} & a_{22} & a_{23} & a_{24} \\
    a_{31} & a_{32} & a_{33} & a_{34} \\
    a_{41} & a_{42} & a_{43} & a_{44} 
\end{array}\right|,
M_{23}=\left|
\begin{array}{ccc} 
    a_{11} & a_{12} & a_{14} \\
    a_{31} & a_{32} & a_{34} \\
    a_{41} & a_{42} & a_{44} 
\end{array}\right|,A_{23}=(-1)^{2+3}M_{23}=-M_{23}
$$
{\color{red}定义:}在$n$阶行列式$D$中,去掉元素$a_{ij}$所在的第$i$行和$j$列,剩下的$n-1$阶行列式被称为元素$a_{ij}$在$D$中的余子式,记作$M_{ij}$。
称$A_{ij}=(-1)^{i+j}M_{ij}$为元素$a_{ij}$在$D$中的代数余子式。\\
{\color{red}定义:}在$n$阶行列式$D=|a_{ij}|_{n}$等于它的任意一行(列)的所有元素与其他代数余子式的乘积之和,即
$$
D=a_{i1}A_{i1}+a_{i2}A_{i2}+\cdots+a_{in}A_{in}
$$
或
$$
D=a_{1j}A_{1j}+a_{2j}A_{2j}+\cdots+a_{nj}A_{nj}
$$
其中$i,j$可以取$1,2,\cdots,n$中任一数值\\
{\color{blue}推论:}若行列式某行(列)的元素全为零,则行列式的值为零。\\
{\color{green}结论:下三角行列式}
$$
\left|\begin{array}{cccc} 
    a_{11} & 0 & \cdots & 0 \\
    a_{21} & a_{22} & \cdots & 0 \\
    \vdots & \vdots & \ddots & \cdots \\
    a_{n1} & a_{n2} & \cdots & a_{nn} 
\end{array}\right|=a_{11}a_{22}\cdots a_{nn}.
$$
{\color{green}结论:上三角行列式}
$$
\left|\begin{array}{cccc} 
    a_{11} & a_{12} & \cdots & a_{1n} \\
    0 & a_{22} & \cdots & a_{2n} \\
    \vdots & \vdots & \ddots & \cdots \\
    0 & 0 & \cdots & a_{nn} 
\end{array}\right|=a_{11}a_{22}\cdots a_{nn}.
$$
{\color{green}结论:对角行列式}
$$
\left|\begin{array}{cccc} 
    a_{11} & 0 & \cdots & 0 \\
    0 & a_{22} & \cdots & 0 \\
    \vdots & \vdots & \ddots & \cdots \\
    0 & 0 & \cdots & a_{nn}
\end{array}\right|=a_{11}a_{22}\cdots a_{nn}.
$$
\section{行列式的性质}
\subsection{行列式的性质}
{\color{red}定义:}称将行列式$D$中的行列互换所得的新的行列式为$D$的转置,记作$D^{T}$\\
例
$$
D=\left|
\begin{array}{ccc}
    1 & 2 & 3 \\
    4 & 5 & 6 \\
    7 & 8 & 9
\end{array}
\right|,
D^{T}=\left|
\begin{array}{ccc}
    1 & 4 & 7 \\
    2 & 5 & 8 \\
    3 & 6 & 9
\end{array}
\right|
$$
即$D^{T}$亦可以视为由$D$以主对角线为轴旋转$180^{\circ}$而得\\
{\color{green}结论:}二阶行列式与它的转置相等。由二阶行列式与它的转置相等可以推出三阶行列式也与它的转置相等 \\
{\color{blue}性质1:}行列式与它的转置相等,即$D=D^{T}$\\
{\color{red}注:}性质1说明行列式中行与列的地位是对等的。因此,凡是对行成立的性质也对列成立\\
{\color{blue}性质2:}交换行列式的两行(列),行列式的值变号(展开后可以用数学归纳法证得)。\\
$$
D=
\left|
\begin{array}{ccc}
    a & b & c \\
    u & v & w \\
    x & y & z
\end{array}\right|
\xlongequal{r_{2}\leftrightarrow r_{3}}
-\left|
\begin{array}{ccc}
    a & b & c \\
    x & y & z \\
    u & v & w
\end{array}\right|=D_{1}
$$
{\color{blue}推论:}如果行列式有两行(列)完全相同,则此行列式等于零 \\
{\color{blue}性质3:}如果行列式的某一行(列)中所有元素有公因子, 则公因子可以提到行列式符号的外面, 即
$$
\left|
\begin{array}{cccc}
    a_{11} & a_{12} & \cdots & a_{1n} \\
    \vdots & \vdots & \ddots & \vdots \\
    ka_{i1} & ka_{i2} & \cdots & ka_{in} \\
    \vdots & \vdots & \ddots & \vdots \\
    a_{n1} & a_{n2} & \cdots & a_{nn} 
\end{array}\right|=
k\left|
\begin{array}{cccc}
    a_{11} & a_{12} & \cdots & a_{1n} \\
    \vdots & \vdots & \ddots & \vdots \\
    a_{i1} & a_{i2} & \cdots & a_{in} \\
    \vdots & \vdots & \ddots & \vdots \\
    a_{n1} & a_{n2} & \cdots & a_{nn} 
\end{array}\right|
$$
{\color{blue}推论:}如果行列式有两行(列)的对应元素成比例, 则行列式的值等于零。\\
{\color{blue}性质4:}若行列式的某一列(行)的元素都是两数之和\\
即,如果
$$
D=\left|
\begin{array}{cccccc}
    a_{11} & a_{12} & \cdots & {\color{red} a_{1j}}+{\color{blue} a^{'}_{1j}} & \cdots & a_{1n}\\
    a_{21} & a_{22} & \cdots & {\color{red} a_{2j}}+{\color{blue} a^{'}_{2j}} & \cdots & a_{2n}\\
    \vdots & \vdots & \ddots & \vdots            & \ddots & \vdots \\
    a_{n1} & a_{n2} & \cdots & {\color{red} a_{nj}}+{\color{blue} a^{'}_{nj}} & \cdots & a_{nn}\\
\end{array}\right|
$$
则$D$等于下列两个行列式之和
$$
\left|
\begin{array}{cccccc}
    a_{11} & a_{12} & \cdots & {\color{red} a_{1j}} & \cdots & a_{1n}\\
    a_{21} & a_{22} & \cdots & {\color{red} a_{2j}} & \cdots & a_{2n}\\
    \vdots & \vdots & \ddots & \vdots            & \ddots & \vdots \\
    a_{n1} & a_{n2} & \cdots & {\color{red} a_{nj}} & \cdots & a_{nn}\\
\end{array}\right|
+\left|
\begin{array}{cccccc}
    a_{11} & a_{12} & \cdots & {\color{blue} a^{'}_{1j}} & \cdots & a_{1n}\\
    a_{21} & a_{22} & \cdots & {\color{blue} a^{'}_{2j}} & \cdots & a_{2n}\\
    \vdots & \vdots & \ddots & \vdots            & \ddots & \vdots \\
    a_{n1} & a_{n2} & \cdots & {\color{blue} a^{'}_{nj}} & \cdots & a_{nn}\\
\end{array}\right|
$$
{\color{red}注:}一次只能拆{\color{red}一行}或{\color{red}一列} \\
{\color{blue}性质5:}把行列式的某一列(行)的各元素乘以同一数k后加到另一列(行)对应的元素上去, 行列式的值不变。\\
$$
\left|
\begin{array}{ccccccc}
    a_{11} & \cdots & {\color{blue} a_{1i}} & \cdots & {\color{red} a_{1j}} & \cdots & a_{1n} \\
    a_{21} & \cdots & {\color{blue} a_{2i}} & \cdots & {\color{red} a_{2j}} & \cdots & a_{2n} \\
    \vdots & \cdots & \vdots & \ddots & \vdots                & \ddots & \ddots \\
    a_{n1} & \cdots & {\color{blue} a_{ni}} & \cdots & {\color{red} a_{nj}} & \cdots & a_{nn} \\
\end{array}\right|
\xlongequal{c_{i}+kc_{j}}
\left|
\begin{array}{ccccccc}
    a_{11} & \cdots & ({\color{blue} a_{1i}}+{\color{red} ka_{1j}}) & \cdots & a_{1j} & \cdots & a_{1n} \\
    a_{21} & \cdots & ({\color{blue} a_{2i}}+{\color{red} ka_{2j}}) & \cdots & a_{2j} & \cdots & a_{2n} \\
    \vdots & \cdots & \vdots                                     & \ddots & \vdots & \ddots & \ddots \\
    a_{n1} & \cdots & ({\color{blue} a_{ni}}+{\color{red} ka_{nj}}) & \cdots & a_{nj} & \cdots & a_{nn} \\
\end{array}\right|
$$
\subsection{行列式的计算}

{\color{blue}行列式的计算方法1:} 利用性质将行列式化为三角行列式(特殊行列式)。\\
{\color{red}注:}这种方法是计算机程序计算行列式的一种常用的方法。利用该方法计算$n$阶行列式大约需要$\frac{2n^{3}}{3}$次运算,在不到一秒钟内就可以计算一个25阶的行列式\\
{\color{blue}计算行列式的方法2(主要方法):}利用性质和展开公式。 \\
{\color{blue}基本思路:}
\begin{enumerate}
    \item{选择一列(行), 利用性质5将该列(行)化出较多的零。}
    \item{利用展开定理将行列式按该列(行)展开。}
    \item{重复以上两步操作。}
\end{enumerate}
{\color{blue}技巧1:}选择数字简单的一行(列)\\
{\color{blue}技巧2:}如果某行(列)只有1或2个元素$\neq 0$, 可按该行(列)直接展开。\\
{\color{blue}技巧3:}如果每行和列都只有2个元素不等于0, 一般按第1行(列)或最后一行(列)直接展开。\\
{\color{blue}技巧4:}行和行列式的计算
$$
D=\left|
\begin{array}{cccc}
    1   & -1  & 1   & x-1 \\
    1   & -1  & x+1 & -1  \\    
    1   & x-1 & 1   & -1  \\    
    x+1 & -1  & 1   & -1  \\
\end{array}\right|
\underset{{{c}_{1}}+{{c}_{4}}}{\overset{
    \begin{smallmatrix} 
    {{c}_{1}}+{{c}_{2}} \\ 
    {{c}_{1}}+{{c}_{3}} 
   \end{smallmatrix}}{\mathop{===}}}\left| 
   \begin{matrix}
      x & -1 & 1 & x-1  \\
      x & -1 & x+1 & -1  \\
      x & x-1 & 1 & -1  \\
      x & -1 & 1 & -1  \\
   \end{matrix} \right|
\underset{{{r}_{4}}-{{r}_{1}}}{\overset{
\begin{smallmatrix} 
{{r}_{2}}-{{r}_{1}} \\ 
{{r}_{3}}-{{r}_{1}} 
\end{smallmatrix}}{\mathop{===}}}
\left| 
\begin{matrix}
    x & -1 & 1 & x-1  \\
    0 & 0 & x & -x  \\
    0 & x & 0 & -x  \\
    0 & 0 & 0 & -x  \\
\end{matrix} \right|
$$$$
=x\times (-1)^{1+1} \times \left|
\begin{array}{ccc}
    {\color{yellow} 0} & x & -x \\
    {\color{yellow} x} & 0 & -x \\    
    {\color{yellow} 0} & 0 & -x \\
\end{array}\right|=x\times x \times (-1)^{2+1}\times\left|
\begin{array}{cc}
    x & -x \\
    0 & -x \\    
\end{array}\right|=x^{4}
$$
{\color{blue} 例:}
$$
D=\left| \begin{matrix}
    a & b & b & \cdots  & b  \\
    b & a & b & \cdots  & b  \\
    b & b & a & \cdots  & b  \\
    \vdots  & \vdots  & \vdots  & \ddots & \vdots   \\
    b & b & b & \cdots  & a  \\
 \end{matrix} \right|
 \underset{\begin{smallmatrix} 
    \cdots  \\ 
    {{c}_{1}}+{{c}_{n}} 
   \end{smallmatrix}}{\overset{\begin{smallmatrix} 
    {{c}_{1}}+{{c}_{2}} \\ 
    {{c}_{1}}+{{c}_{3}} 
   \end{smallmatrix}}{\mathop{===}}}  
\left| \begin{matrix}
    a+(n-1)b & b & b    & \cdots  & b  \\
    a+(n-1)b & a & b    & \cdots  & b  \\
    a+(n-1)b & b & a    & \cdots  & b  \\
    \vdots   & \vdots   & \vdots  & \ddots & \vdots   \\
    a+(n-1)b & b & b    & \cdots  & a  \\
\end{matrix} \right| 
$$$$
D=\underset{\begin{smallmatrix} 
    \cdots  \\ 
    {{r}_{n}}-{{r}_{1}} 
   \end{smallmatrix}}{\overset{\begin{smallmatrix} 
    {{r}_{2}}-{{r}_{1}} \\ 
    {{r}_{3}}-{{r}_{1}} 
   \end{smallmatrix}}{\mathop{===}}}\,\ \left| \begin{matrix}
      a+(n-1)b & b & b & \cdots  & b  \\
      0 & a-b & 0 & \cdots  & 0  \\
      0 & 0 & a-b & \cdots  & 0  \\
      \vdots  & \vdots  & \vdots  & \ddots & \vdots   \\
      0 & 0 & 0 & \cdots  & a-b  \\
   \end{matrix} \right|
   =[a+(n-1)b]{{(a-b)}^{n-1}}
$$
\subsection{行列式的性质(II)}
{\color{blue}例:}试计算范德蒙德 (Vandermonde)行列式
$$
D=\left| \begin{matrix}
    1 & 1 & 1 & 1  \\
    {{x}_{1}} & {{x}_{2}} & {{x}_{3}} & {{x}_{4}}  \\
    x_{1}^{2} & x_{2}^{2} & x_{3}^{2} & x_{4}^{2}  \\
    x_{1}^{3} & x_{2}^{3} & x_{3}^{3} & x_{4}^{3}  \\
 \end{matrix} \right|
 \underset{{{r}_{2}}-{{x}_{1}}{{r}_{1}}}{\overset{\begin{smallmatrix} 
    {{r}_{4}}-{{x}_{1}}{{r}_{3}} \\ 
    {{r}_{3}}-{{x}_{1}}{{r}_{2}} 
   \end{smallmatrix}}{\mathop{====}}}
\left| \begin{matrix}
    1 & 1 & 1 & 1  \\
    0 & {{x}_{2}}-{{x}_{1}} & {{x}_{3}}-{{x}_{1}} & {{x}_{4}}-{{x}_{1}}  \\
    0 & x_{2}^{2}-{{x}_{1}}{{x}_{2}} & x_{3}^{2}-{{x}_{1}}{{x}_{3}} & x_{4}^{2}-{{x}_{1}}{{x}_{4}}  \\
    0 & x_{2}^{3}-{{x}_{1}}x_{2}^{2} & x_{3}^{3}-{{x}_{1}}x_{3}^{2} & x_{4}^{3}-{{x}_{1}}x_{4}^{2}  \\
 \end{matrix} \right|
$$
{\color{blue}技巧5:}逐行相减(相加)
$$
\underset{\mbox{展开}}{\overset{\mbox{按第一列}}{\mathop{====}}}\,
\left| \begin{matrix}
    {{x}_{2}}-{{x}_{1}} & {{x}_{3}}-{{x}_{1}} & {{x}_{4}}-{{x}_{1}}  \\
    x_{2}^{2}-{{x}_{1}}{{x}_{2}} & x_{3}^{2}-{{x}_{1}}{{x}_{3}} & x_{4}^{2}-{{x}_{1}}{{x}_{4}}  \\
    x_{2}^{3}-{{x}_{1}}x_{2}^{2} & x_{3}^{3}-{{x}_{1}}x_{3}^{2} & x_{4}^{3}-{{x}_{1}}x_{4}^{2}  \\
 \end{matrix} \right|
\underset{\mbox{公因子}}{\overset{\mbox{提取每列}}{\mathop{====}}}\,
 ({{x}_{2}}-{{x}_{1}})({{x}_{3}}-{{x}_{1}})({{x}_{4}}-{{x}_{1}})\left| \begin{matrix}
    1 & 1 & 1  \\
    {{x}_{2}} & {{x}_{3}} & {{x}_{4}}  \\
    x_{2}^{2} & x_{3}^{2} & x_{4}^{2}  \\
 \end{matrix} \right|
$$
$$
=({{x}_{2}}-{{x}_{1}})({{x}_{3}}-{{x}_{1}})({{x}_{4}}-{{x}_{1}})({{x}_{3}}-{{x}_{2}})({{x}_{4}}-{{x}_{2}})\left| \begin{matrix}
    1 & 1  \\
    {{x}_{3}} & {{x}_{4}}  \\
 \end{matrix} \right| 
$$
{\color{blue}技巧6:}数学归纳法(递推公式)
$$
D_{n}=\left| \begin{matrix}
    1 & 1 & \cdots  & 1  \\
    {{x}_{1}} & {{x}_{2}} & \cdots  & {{x}_{n}}  \\
    x_{1}^{2} & x_{2}^{2} & \cdots  & x_{n}^{2}  \\
    \vdots  & \vdots  & {} & \vdots   \\
    x_{1}^{n-1} & x_{2}^{n-1} & \cdots  & x_{n}^{n-1}  \\
 \end{matrix} \right|
 =\prod\limits_{1\le j<i\le n}{({{x}_{i}}-{{x}_{j}})} 
$$
$$
=({{x}_{n}}-{{x}_{1}})({{x}_{n-1}}-{{x}_{1}})\cdots({{x}_{3}}-{{x}_{1}})({{x}_{2}}-{{x}_{1}})
$$$$
({{x}_{n}}-{{x}_{2}})({{x}_{n-1}}-{{x}_{2}})\cdots 
({{x}_{3}}-{{x}_{2}}) \cdots
({{x}_{n}}-{{x}_{n-2}})({{x}_{n-1}}-{{x}_{n-2}}) \\ 
({{x}_{n}}-{{x}_{n-1}}) \\ 
$$  
{\color{blue}技巧7:}利用性质化为特殊行列式
$$
\left| \begin{matrix}
    a & b & c  \\
    {{a}^{2}} & {{b}^{2}} & {{c}^{2}}  \\
    b+c & c+a & a+b  \\
 \end{matrix} \right|
 \overset{{{r}_{3}}+{{r}_{1}}}{\mathop{===}}\,\left| \begin{matrix}
    a & b & c  \\
    {{a}^{2}} & {{b}^{2}} & {{c}^{2}}  \\
    a+b+c & a+b+c & a+b+c  \\
 \end{matrix} \right|
 \underset{\mbox{公因子}}{\overset{\mbox{提取}}{\mathop{===}}}\,(a+b+c)\left| \begin{matrix}
    a & b & c  \\
    {{a}^{2}} & {{b}^{2}} & {{c}^{2}}  \\
    1 & 1 & 1  \\
 \end{matrix} \right|
 $$$$
\underset{{{r}_{2}}\leftrightarrow{{r}_{1}}}{\overset{{{r}_{3}}\leftrightarrow {{r}_{2}}}{\mathop{===}}}\,(a+b+c)\left| \begin{matrix}
    1 & 1 & 1  \\
    a & b & c  \\
    {{a}^{2}} & {{b}^{2}} & {{c}^{2}}  \\
 \end{matrix} \right|
 =(a+b+c)(b-a)(c-a)(c-b)
$$
{\color{blue}结论:}
$$
D={{\left| \begin{matrix}
    {{a}_{11}} & \cdots  & {{a}_{1m}} & 0 & \cdots  & 0  \\
    \vdots  & {m\times n} & \vdots  & \vdots  & {0} & \vdots   \\
    {{a}_{m1}} & \cdots  & {{a}_{mm}} & 0 & \cdots  & 0  \\
    {{c}_{11}} & \cdots  & {{c}_{1m}} & {{b}_{11}} & \cdots  & {{b}_{1n}}  \\
    \vdots  & {*} & \vdots  & \vdots  & {n\times n} & \vdots   \\
    {{c}_{n1}} & \cdots  & {{c}_{nm}} & {{b}_{n1}} & \cdots  & {{b}_{nn}}  \\
 \end{matrix} \right|}_{m+n}}
=\left|
\begin{matrix}
    a_{11} & \cdots  & a_{1m}  \\
    \vdots  & \ddots & \vdots   \\
    a_{m1} & \cdots  & a_{mm}
\end{matrix}\right|_{m}
\left|\begin{matrix}
    b_{11} & \cdots  & b_{1n}  \\
    \vdots  & \ddots & \vdots   \\
    b_{n1} & \cdots  & b_{nn} \\
\end{matrix} \right|_{n}
$$
\section{克莱姆法则(cramer)}
{\color{red}定义:}包含未知量$x_{1},x_{2},\cdots,x_{n}$的形如
$$a_{1}x_{1}+a_{2}x_{2}+\cdots+a_{n}x_{n}=b$$
的方程称为线性方程,形如
$$
\left\{ 
\begin{aligned}
    {{a}_{11}}{{x}_{1}}+{{a}_{12}}{{x}_{2}}+\cdots +{{a}_{1n}}{{x}_{n}}={{b}_{1}},  \\
    {{a}_{21}}{{x}_{1}}+{{a}_{22}}{{x}_{2}}+\cdots +{{a}_{2n}}{{x}_{n}}={{b}_{2}},  \\
    \cdots \cdots \cdots \cdots \cdots \cdots \cdots \cdots \cdots \cdots   \\
    {{a}_{m1}}{{x}_{1}}+{{a}_{m2}}{{x}_{2}}+\cdots +{{a}_{mn}}{{x}_{n}}={{b}_{m}}  \\
\end{aligned}
\right.
$$
{\color{red}定义:}对于线性方程组
\[\left\{ \begin{matrix}
    {{a}_{11}}{{x}_{1}}+{{a}_{12}}{{x}_{2}}+\cdots +{{a}_{1n}}{{x}_{n}}={{b}_{1}};  \\
    {{a}_{21}}{{x}_{1}}+{{a}_{22}}{{x}_{2}}+\cdots +{{a}_{2n}}{{x}_{n}}={{b}_{2}};  \\
    \cdots \cdots  \cdots \cdots \cdots \cdots \cdots \cdots \cdots \cdots   \\
    {{a}_{n1}}{{x}_{1}}+{{a}_{n2}}{{x}_{2}}+\cdots +{{a}_{nn}}{{x}_{n}}={{b}_{n}},  \\
 \end{matrix} \right.\]
 称行列式
 $$
 D=\left|\begin{matrix}
    a_{11} & a_{12} & \cdots  & a_{1n}  \\
    a_{21} & a_{22} & \cdots  & a_{2n}  \\
    \vdots & \vdots & \ddots & \vdots   \\
    a_{n1} & a_{n2} & \cdots  & a_{nn}  \\
\end{matrix} \right|
 $$
 为方程组的系数行列式。\\
 {\color{blue}定理:}如果线性方程组的系数行列式不等于0, 即
 $$
 D=\left| \begin{matrix}
    \begin{matrix}
    {{a}_{11}} & {{a}_{12}} & \cdots  & {{a}_{1n}}  \\
 \end{matrix}  \\
    \begin{matrix}
    {{a}_{21}} & {{a}_{22}} & \cdots  & {{a}_{2n}}  \\
 \end{matrix}  \\
    \vdots \text{   }\vdots \text{      }\vdots   \\
    \begin{matrix}
    {{a}_{n1}} & {{a}_{n2}} & \cdots  & {{a}_{nn}}  \\
 \end{matrix}  \\
 \end{matrix} \right|\ne 0 
 $$
 那么线性方程组有惟一解, 并且解可以表示为
$$
{{x}_{1}}=\frac{{{D}_{1}}}{D},
{{x}_{2}}=\frac{{{D}_{2}}}{D},
{{x}_{3}}=\frac{{{D}_{3}}}{D},\cdots,
{{x}_{n}}=\frac{{{D}_{n}}}{D}
$$
其中
$$
{{D}_{j}}=\left| \begin{matrix}
    {{a}_{11}} & \cdots  & {{a}_{1,j-1}} & {\color{yellow}{{b}_{1}}} & {{a}_{1,j+1}} & \cdots  & {{a}_{1n}}  \\
    \vdots  & {\ddots} & \vdots          & {\color{yellow}\vdots}  & \vdots  & {\ddots} & \vdots   \\
    {{a}_{n1}} & \cdots  & {{a}_{n,j-1}} & {\color{yellow}{{b}_{n}}} & {{a}_{n,j+1}} & \cdots  & {{a}_{nn}}  \\
 \end{matrix} \right| 
$$
{\color{red}注:}如果线性方程组的系数行列式$D=0$则方程组无解或有无穷多解。其中$b_{1},\cdots,b_{n}$为方程组的等号右边部分行成的向量\\
称线性方程组
$$
\left\{ \begin{matrix}
    {{a}_{11}}{{x}_{1}}+{{a}_{12}}{{x}_{2}}+\cdots +{{a}_{1n}}{{x}_{n}}={\color{red}0}  \\
    {{a}_{21}}{{x}_{1}}+{{a}_{22}}{{x}_{2}}+\cdots +{{a}_{2n}}{{x}_{n}}={\color{red}0}  \\
    \cdots \cdots \cdots \cdots \cdots \cdots \cdots \cdots \cdots \cdots   \\
    {{a}_{m1}}{{x}_{1}}+{{a}_{m2}}{{x}_{2}}+\cdots +{{a}_{mn}}{{x}_{n}}={\color{red}0}  \\
 \end{matrix} \right. 
$$
为{\color{red}齐次}线性方程组。显然$x_{1}=0,x_{2}=0,\cdots,x_{n}=0$是线性方程组的一个解,称为{\color{blue}零解}。\\
{\color{red}注:}齐次线性方程组的解只有两种情况:惟一解或无穷多解。和非齐次线性方程组不同,它没有“无解”这
种情况。\\
{\color{blue}推论:}如果齐次线性方程组
$$
\left\{ \begin{matrix}
    {{a}_{11}}{{x}_{1}}+{{a}_{12}}{{x}_{2}}+\cdots +{{a}_{1n}}{{x}_{n}}={\color{red}0}  \\
    {{a}_{21}}{{x}_{1}}+{{a}_{22}}{{x}_{2}}+\cdots +{{a}_{2n}}{{x}_{n}}={\color{red}0}  \\
    \cdots \cdots \cdots \cdots \cdots \cdots \cdots \cdots \cdots \cdots   \\
    {{a}_{m1}}{{x}_{1}}+{{a}_{m2}}{{x}_{2}}+\cdots +{{a}_{mn}}{{x}_{n}}={\color{red}0}  \\
 \end{matrix} \right. 
$$
的系数行列式$D\ne 0$,  则方程组只有零解。
{\color{red}注:} 如果齐次线性方程组的系数行列式$D=0$, 则方程组必有非零解(即有无穷多解)。\\



  



