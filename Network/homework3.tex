\documentclass{article}
\usepackage{ctex}
\usepackage{fancyhdr}
\usepackage{extramarks}
\usepackage{amsmath}
\usepackage{amsthm}
\usepackage{amsfonts}
\usepackage{tikz}
\usepackage[plain]{algorithm}
\usepackage{algpseudocode}

\usetikzlibrary{automata,positioning}

%
% Basic Document Settings
%

\topmargin=-0.45in
\evensidemargin=0in
\oddsidemargin=0in
\textwidth=6.5in
\textheight=9.0in
\headsep=0.25in

\linespread{1.1}

\pagestyle{fancy}
\lhead{\hmwkAuthorName}
\chead{\hmwkClass\ (\hmwkClassInstructor\ \hmwkClassTime): \hmwkTitle}
\rhead{\firstxmark}
\lfoot{\lastxmark}
\cfoot{\thepage}

\renewcommand\headrulewidth{0.4pt}
\renewcommand\footrulewidth{0.4pt}

\setlength\parindent{0pt}

%
% Create Problem Sections
%

\newcommand{\enterProblemHeader}[1]{
    \nobreak\extramarks{}{Problem \arabic{#1} continued on next page\ldots}\nobreak{}
    \nobreak\extramarks{Problem \arabic{#1} (continued)}{Problem \arabic{#1} continued on next page\ldots}\nobreak{}
}

\newcommand{\exitProblemHeader}[1]{
    \nobreak\extramarks{Problem \arabic{#1} (continued)}{Problem \arabic{#1} continued on next page\ldots}\nobreak{}
    \stepcounter{#1}
    \nobreak\extramarks{Problem \arabic{#1}}{}\nobreak{}
}

\setcounter{secnumdepth}{0}
\newcounter{partCounter}
\newcounter{homeworkProblemCounter}
\setcounter{homeworkProblemCounter}{1}
\nobreak\extramarks{Problem \arabic{homeworkProblemCounter}}{}\nobreak{}

\newenvironment{homeworkProblem}{
    \section{Problem \arabic{homeworkProblemCounter}}
    \setcounter{partCounter}{1}
    \enterProblemHeader{homeworkProblemCounter}
}{
    \exitProblemHeader{homeworkProblemCounter}
}

%
% Homework Details
%   - Title
%   - Due date
%   - Class
%   - Section/Time
%   - Instructor
%   - Author
%

\newcommand{\hmwkTitle}{第三章}
\newcommand{\hmwkDueDate}{\today}
\newcommand{\hmwkClass}{计算机网络}
\newcommand{\hmwkClassTime}{}
\newcommand{\hmwkClassInstructor}{}
\newcommand{\hmwkAuthorName}{范李勇}

% \title{计算机网络第三章课后习题}
% \author{\kaishu 范李勇}
% \date{\today}



%
% Title Page
%

\title{
    \vspace{2in}
    \textmd{\textbf{\hmwkClass:\ \hmwkTitle}}\\
    \normalsize\vspace{0.1in}\small{Due\ on\ \hmwkDueDate}\\
    \vspace{0.1in}\large{\textit{\hmwkClassInstructor\ \hmwkClassTime}}
    \vspace{3in}
}
\author{\textbf{\hmwkAuthorName}}
\date{}

\renewcommand{\part}[1]{\textbf{\large Part \Alph{partCounter}}\stepcounter{partCounter}\\}

%
% Various Helper Commands
%

% Useful for algorithms
\newcommand{\alg}[1]{\textsc{\bfseries \footnotesize #1}}

% For derivatives
\newcommand{\deriv}[1]{\frac{\mathrm{d}}{\mathrm{d}x} (#1)}

% For partial derivatives
\newcommand{\pderiv}[2]{\frac{\partial}{\partial #1} (#2)}

% Integral dx
\newcommand{\dx}{\mathrm{d}x}

% Alias for the Solution section header
\newcommand{\solution}{\textbf{\large Solution}}

% Probability commands: Expectation, Variance, Covariance, Bias
\newcommand{\E}{\mathrm{E}}
\newcommand{\Var}{\mathrm{Var}}
\newcommand{\Cov}{\mathrm{Cov}}
\newcommand{\Bias}{\mathrm{Bias}}

\begin{document}

\maketitle

\pagebreak

\begin{homeworkProblem}
    {\Large \textbf{3-02 数据链路层中的链路控制包括哪些功能?试讨论数据链路层做成可靠的链路层有哪些优点和缺点}}\\
    \begin{itemize}
        \item{(1)	链路管理}
        \item{(2) 帧定界 }
        \item{(3) 流量控制 }
        \item{(4) 差错控制 }
        \item{(5) 将数据和控制信息区分开 }
        \item{(6) 透明传输 }
        \item{(7) 寻址}
    \end{itemize}
    可靠的链路层的优点和缺点取决于所应用的环境:对于干扰严重的信道,可靠的链路层可以将重传范围约束在局部链路,,防止全网络的传输效率受损,对于优质信道,采用可靠的链路层会增大资源开销,影响传输效率。
\end{homeworkProblem}
\begin{homeworkProblem}
    {\Large \textbf{3-03网络适配器的作用是什么?网络适配器工作在哪一层?}}\\
    适配器来实现数据链路层和物理层这两层的协议的硬件和软件
网络适配器工作在TCP/IP协议中的网络接口层
\end{homeworkProblem}
\begin{homeworkProblem}
    {\Large \textbf{3-07要发送的数据为1101011011。采用CRC的生成多项式是$P(X)=X^4+X+1$。试求应添加在数据后面的余数。数据在传输过程中最后一个1变成了0,问接收端能否发现?}}\\
    \begin{tabular}{cc}
        $\quad \quad \quad \quad \ 110000101$\\     
        $10011\overline{)11010110110000}$\\
        $\overline{10011}$\\
        $10011$\\
        $\quad \quad \quad \quad \overline{10110}$\\
        $\quad \quad \quad \quad 10011$\\
        $\quad \quad \quad \quad \quad \quad \overline{10100}$\\
        $\quad \quad \quad \quad \quad \quad 10011$\\
        $\quad \quad \quad \quad \quad \quad \quad \quad \overline{1110}$
    \end{tabular}
\end{homeworkProblem}
\begin{homeworkProblem}
    {\Large \textbf{3-20假定1km长的CSMA/CD网络的数据为1Gbit/s。设信号在网络上的传播速率为200000km/s。求能够使用此协议的最短帧长。}}\\
    对于1km电缆,单程传播时间为1/200000=5微秒,来回路程传播时间为10微秒,为了能够按照CSMA/CD工作,最小帧的发射时间不能小于10微秒。
$$\frac{10\times 10^{-6}}{1\times 10^{-9}}=10000$$
    所以最短帧是10000位或1250字节长。
\end{homeworkProblem}
\begin{homeworkProblem}
    {\Large \textbf{3-28 10Mbit/s以太网升级到100Mbit/s、1Gbit/s和10Gbit/s时,都需要解决哪些技术问题?为什么以太网能够在发展过程中淘汰掉自己的竞争对手,并使自己的应用范围从局域网一直扩展到城域网和广域网?}}\\
    以太网升级时,由于数据传输率提高了,帧的发送时间会按比例缩短,这样会影响冲突的检测。所以需要减小最大电缆长度或增大帧的最小长度,使参数a保持为较小的值,才能有效地检测冲突。以太网是一种经过实践证明成熟的技术,价格便宜,易于安装、使用,提供不同的传输组率,可满足不同的需求。
\end{homeworkProblem}
\begin{homeworkProblem}
    {\Large \textbf{3-32假定在图3-30中的所有链路的速率仍为100Mbit/s,但所有的以太网交换机都换成为100Mbit/s的集线器。试计算这9台主机和两个服务器产生的总的吞吐量的最大值。为什么?}}\\
    现在整个系统是一个碰撞域,因此其最大吞吐量为100 Mbit/s。
\end{homeworkProblem}
\end{document}
