%导言区
\documentclass[12pt]{report}%book、report、letter、article
%在文档区可使用可选参数设置normal大小只有10、11、12
\usepackage{ctex}%使用中文引入包可以使用将前面头一行修改为\documentclass{ctexart}代替
%详情可见终端输入texdoc ctex查看
%文件源码格式需要更改为utf-8才能使用显示中文,并需要使用xelatex来编译本文档
\usepackage{amsmath}
\title{\heiti 嵌入式作业\footnote{edit by \LaTeX}}
\author{\kaishu 范李勇}
\date{\today}

%正文区
\begin{document}
\maketitle
\section*{1、嵌入式系统中特殊功能寄存器有哪些?分别用于什么功能?} 
\subsection*{(1)R13 ——> SP,为堆栈寄存器,用于C语言类程序之间调用所需的空间指针}
\subsection*{(2)R14 ——> LR,为连接寄存器,在发生程序调用时,一般用户存放程序返回地址}
\subsection*{(3)R15 ——> PC,为程序计数寄存器,存放西一条要执行的程序码地址}
\section*{2、请写出0x21436587在大端和小端模式下的存储结构。}
\subsection*{大端模式:内存地址从小到大的存储内容为0x21、0x43、0x65、0x87}
\subsection*{小端模式:内存地址从小到大的存储内容为0x87、0x65、0x43、0x21}
\section*{3、写出嵌入式中子函数调用的过程。}
\subsection*
{\heiti 使用IP(R12)暂时保存栈指针sp,
然后使用堆栈操作指令stmfd将栈帧(FP)、IP、程序返回地址
(LR)、程序计数器(PC)压栈,以保护现场,然后使用
sub fp,ip,\#4使fp指向当前函数栈帧的栈底,
sub sp,sp,\#8,为当前函数局部变量分配看空间。
接下来通过寄存器传递参数r1,r2,r3,r4。
使用BL指令调用函数,
BL指令同时也会将当前指令的下一条指令地址赋给LR,
以跳转回来。最后使用ldmfd恢复现场。}


\end{document}
