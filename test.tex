%导言区
\documentclass[12pt]{book}%book、report、letter、article
%在文档区可使用可选参数设置normal大小只有10、11、12
\usepackage{ctex}%使用中文引入包可以使用将前面头一行修改为\documentclass{ctexart}代替
%详情可见终端输入texdoc ctex查看
%文件源码格式需要更改为utf-8才能使用显示中文,并需要使用xelatex来编译本文档
\usepackage{amsmath}
\title{\heiti 线性代数笔记\footnote{edit by \LaTeX}}
\author{\kaishu 范李勇}
\date{\today}

%正文区
\begin{document}
\maketitle
\chapter{行列式}
\section{ 1、二阶行列式}
定义一个
$$
\left|\begin{array}{cc} a_{11} & a_{12} \\
a_{21} & a_{22} \end{array}\right|
$$
为二阶行列式,主对角线为$a_{11}$和$a_{22}$,副对角线为$a_{12}$和$a_{21}$。并规定
$$
\left|\begin{array}{cc} a_{11} & a_{12} \\
a_{21} & a_{22} \end{array}\right|=a_{11}\times a_{22}-a_{12}\times a_{21}
$$
值相等的两个行列式称为这两个行列式相等。\\
称用$n^2$个元素$a_{ij}  (i,j =1,2,…,n)$ 组成的如下对象
$$
D=\left|
\begin{array}{cccc} 
    a_{11} & a_{12} & \cdots & a_{1n} \\ 
    a_{21} & a_{22} & \cdots & a_{2n} \\ 
    \vdots & \vdots & \ddots & \vdots \\
    a_{n1} & a_{n2} & \cdots & a_{nn} 
\end{array}
\right|=\det(a_{ij})=|a_{ij}|_{n}
$$
为一个n 阶行列式(determinant)。
$$
D=\left|
\begin{array}{cccc} 
    a_{11} & a_{12} & a_{13} & a_{14} \\
    a_{21} & a_{22} & a_{23} & a_{24} \\
    a_{31} & a_{32} & a_{33} & a_{34} \\
    a_{41} & a_{42} & a_{43} & a_{44} 
\end{array}\right|,
M_{23}=\left|
\begin{array}{ccc} 
    a_{11} & a_{12} & a_{14} \\
    a_{31} & a_{32} & a_{34} \\
    a_{41} & a_{42} & a_{44} 
\end{array}\right|,A_{23}=(-1)^{2+3}M_{23}=-M_{23}
$$
定义:在$n$阶行列式$D$中,去掉元素$a_{ij}$所在的第$i$行和$j$列,剩下的$n-1$阶行列式被称为元素$a_{ij}$在$D$中的余子式,记作$M_{ij}$。
称$A_{ij}=(-1)^{i+j}M_{ij}$为元素$a_{ij}$在$D$中的代数余子式。
\end{document}
