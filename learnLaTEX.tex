%导言区
\documentclass[12pt]{article}%book、report、letter、article
%在文档区可使用可选参数设置normal大小只有10、11、12
\usepackage{ctex}%使用中文引入包可以使用将前面头一行修改为\documentclass{ctexart}代替
%详情可见终端输入texdoc ctex查看
%文件源码格式需要更改为utf-8才能使用显示中文,并需要使用xelatex来编译本文档
\usepackage{amsmath}
\newcommand\degree{^\circ}%使用newcommand命令来定义新的命令
\title{\heiti 测试我的 vs code for \LaTeX}%设置标题
\author{\kaishu 张三}%设置作者
\date{\heiti \today}%设置日期 \today是今天
%正文区
\begin{document}
% \maketitle%除了letter都是使用这个把标题作者时间引入到文章当中
\maketitle
测试中文
哈哈哈哈
你好啊
test math $f(x)=x^2+3\degree$ %用单个或两个美元符号皆可使用数学模式区别是两个美元符号会另起一行来显示数学公式
\begin{equation}
    AB^2=BC^2+AC^2
\end{equation}%用于产生带编号的行间公式
\begin{equation}
    \delta A=b*c
\end{equation}
%字体族设置(罗马、无衬线、打字机)

\textrm{Roman Family}%设置字体族为罗马字体
%\textsf{}为无衬线字体族\texttt为打字机字体族

\textsf{Sans Serif Family}

\texttt{Typewriter Family}

{\rmfamily Roman Family 测试中文}
%设置后面字体族为罗马字体
%\sffamily为无衬线字体族\ttfamily为打字机字体族

{\sffamily Sans Serif Family}用大括号分组可声明设置范围

{\ttfamily Typewriter Family 用大括号分组可声明设置范围}
%用大括号分组可声明设置范围

%字体系列设置(粗细、宽度)
\textmd{Medium Series} \textbf{Boldfacae Series}

{\mdseries Medium Series} {\bfseries Boldfacae Series}

%字体形状(直立、斜体、伪斜体、小型大写)
\textup{Upright Shape} \textit{Italic Shape}
\textsl{Slanted Shape} \textsc{Small Caps Shape}

{\upshape Upright Shape} {\itshape Italic Shape}
{\slshape Slanted Shape} {\scshape Small Caps Shape}

%中文字体
{\songti 宋体}
\quad 
{\heiti 黑体} 
\quad 
{\fangsong 仿宋}
\quad 
{\kaishu 楷书} 

中文字体 \textbf{粗体} 与  \textit{斜体}
\\ \\ \\ \\ \\ \\ \\ \\ \\ \\ \\ \\ \\ 
%字体大小
{\tiny  Hello}\\
{\scriptsize Hello}\\
{\footnotesize Hello}\\
{\small Hello}\\
{\normalsize Hello}\\
{\large Hello}\\
{\Large Hello}\\
{\LARGE Hello}\\
{\huge Hello}\\
{\Huge Hello}

\[
  \lim_{x \to 0}^\infty
\]
\end{document}