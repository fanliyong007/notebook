
\chapter*{算法是什么}
\begin{description}
    \item[输入] 待处理的信息(问题)
    \item[输出] 经处理的信息(答案)
    \item[正确性] 的确可以解决指定的问题
    \item[确定性] 任一算法都可以描述为由一个基本操作组成的序列
    \item[可行性] 每一基本操作都可实现,且在常数时间内完成(ps:把大象装进冰箱并不是算法)
    \item[有穷性] 对于任何输入,经有穷次基本操作,都可以得到输出
\end{description}
算法 $\neq$ 程序

好的算法是既要马儿跑又要马儿少吃草

度量
理想、统一、分层次的尺度
正确性
运行时间+所需存储空间

$T_A(P)=$
用算法A求解某一问题规模为n的实例,所需计算成本讨论特定算法A(及其相对应的问题)时,简记$T(n)$

$T(n)= max\{T(P) \ | \ |P|=n\}$ \\
在规模同为n的所有实例中只关注最坏(成本最高)者 \\
$T(n)=$算法为求解规模为n的问题,所需执行的基本操作次数(忽略硬件平台等运用该尺度测量DSA的性能

